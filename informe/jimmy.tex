%--- commands/defs
\newcommand{\img}[2]{\includegraphics[scale=#1]{{#2}.png}}
\newcommand{\jmc}[1]{{\color{red}{(JM: #1)}}}   % jimmy comment

%--- doc
\section{Relaci\'on entre g\'enero y estructura de comunidades}

Para cuantificar la relaci\'on entre el g\'enero de los delfines y la estructura de comunidades que fueron deducidas por los diferentes algoritmos (e.g. {\it Greedy}), empleamos la definici\'on de {\it Informacion Mutua}:

\begin{align}
    I(\{c1\},\{c2\}) = \sum_{c1,c2} p(c1,c2) log \frac{p(c1,c2)}{p(c1) p(c2)},
\label{eq:info_mutua}
\end{align}

donde $c1$ y $c2$ son etiquetas de las comunas del grafo.
Las comunas construidas en el presente grafo fueron deducidas usando los algoritmos {\it greedy, betweenness, infomap} y {\it louvain}.
La definici\'on \ref{eq:info_mutua} cuantifica cuanto se departa la informaci\'on real del grafo, respecto de la cantidad de informaci\'on que brinda una grafo cuyas comunas estan descorrelacionadas del g\'enero. \jmc{che Seba, a ver si esto tiene sentido}


\begin{figure}
    \centering
    \img{0.5}{p12_greedy}
    \img{0.5}{p12_betweenness}
    \img{0.5}{p12_infomap}
    \img{0.5}{p12_louvain}
    \caption{
    Valores de las matrices de probabilidad conjunto para los algoritmos {\it greedy} (izquierda, arriba) {\it betweenness} (derecha, arriba), {\it infomap} (izquierda, abajo) y {\it louvain} (derecha, abajo). 
    }
\label{fig:prob_conj}
\end{figure}


%EOF
