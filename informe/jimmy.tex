%--- commands/defs
\newcommand{\img}[2]{\includegraphics[scale=#1]{{#2}.png}}
\newcommand{\jmc}[1]{{\color{red}{(JM: #1)}}}   % jimmy comment
\newcommand{\cm}[1]{{\color{blue}{(RB \& MC: #1)}}}   % raul & mari comment



%--- doc
\newpage
\section{Relaci\'on entre comunidades}
\label{sec:info_mutua}

%Para cuantificar la relaci\'on entre el g\'enero de los delfines y la estructura de comunidades que fueron deducidas por los diferentes algoritmos (e.g. {\it Greedy}), empleamos la definici\'on de {\it Informacion Mutua}:
%Para cuantificar la relaci\'on entre comunidades de la red, definidas por dos conjuntos etiquetas de la red, $\{c1\}$ y $\{c2\}$, podemos usar la definici\'on de {\it Informacion Mutua}:
A trav\'es del \'indice de \textit{Informaci\'on Mutua} podemos cuantificar la 
similitud entre particiones, de  comunidades de la red, definidas por dos 
conjuntos etiquetas $\{C_1\}$ y $\{C_2\}$. Este est\'a dado por
\begin{align}
    I(\{C_1\},\{C_2\}) = \sum_{C_1 ,C_2} p(C_1,C_2) \log \frac{p(C_1,C_2)}{p(C_1) p(C_2)},
\label{eq:info_mutua}
\end{align}
o su versi\'on normalizada
\begin{align}
    \label{eq:info_mutua_norm}
    I_n (\{C_1\},\{C_2\}) &= \frac{ 2 I(\{C_1\},\{C_2\}) }{  H(\{C_1\}) + H(\{C_2\}) }  \\
    \intertext{donde}
    H(C) &= - \sum_{c_i \in C} p(c_i) \log(p(c_i))
\end{align}
es la informaci\'on total de la partici\'on $C\equiv \{c_i\}$.



Las comunas construidas en el presente grafo fueron deducidas usando los algoritmos {\it greedy, betweenness, infomap} y {\it louvain} \cm{Esto me parece que que es redundante\ldots y el siguiente parrafo no lo entendemos}.
La definici\'on \ref{eq:info_mutua} cuantifica cuanto se departa la informaci\'on real del grafo, respecto de la cantidad de informaci\'on que brinda una grafo cuyas comunas estan descorrelacionadas del g\'enero. \jmc{che Seba, a ver si esto tiene sentido}.
%--- valor de In en dos casos extremos
En el caso particular en que los conjuntos $\{C_1\}$ y $\{C_2\}$ representen variable independientes, entonces se dice que el conjunto $\{C_1\}$ no da brinda ninguna informaci\'on sobre el conjunto $\{C_2\}$, y de acuerdo a la ec. \ref{eq:info_mutua_norm} obtenemos $I_n=0$.
Y en el caso particular en que $\{C_1\}$ y $\{C_2\}$ son el mismo conjunto, obtenemos la informaci\'on mutua normalizada $I_n=1$.

\subsection{Comparaci\'on entre algoritmos de reconocimiento de comunidades}

La cuantificaci\'on de {informaci\'on dada por la Ec. \ref{eq:info_mutua} 
consta tanto de: la medici\'on de la probabilidad de que un nodo pertenezca
a una comunidad $C_i$ ($p(C_i)$), como de la probabilidad conjunta de que un
nodo pertenezca a una comunidad $C_i$ en la partici\'on $\{C_i\}$ y pertenezca
a la comunidad $C_j$ en la partici\'on $\{C_j\}$ ($p(C_i,C_j)$). La primera
distribuc\'on de pertenencia a etiquetas/comunidades se puede ver en la 
figura \ref{fig:histos}. En ella se puede observar que el etiquetado muestra 
distintas distribuciones en cada caso, adem\'as es importante notar que 
etiquetados iguales no representan las mismas comunidades entre cada algoritmo,
por lo tanto, no existe una \'unica distribuci\'on que represente cada caso;
Por otro lado, el caso de la probabilidad
conjunta es mostrado en las matrices de la figura \ref{fig:matrix}.


\begin{figure}[!ht]
    \centering
    \begin{subfigure}[b]{.45\columnwidth}
        \includegraphics[width=0.95\columnwidth]{figuras/louvain_probability.pdf}
    \end{subfigure}
    \begin{subfigure}[b]{.45\columnwidth}
        \includegraphics[width=0.95\columnwidth]{figuras/edge_betweeness_probability.pdf}
    \end{subfigure}\\
    \begin{subfigure}[b]{.45\columnwidth}
        \includegraphics[width=0.95\columnwidth]{figuras/infomap_probability.pdf}
    \end{subfigure}
    \begin{subfigure}[b]{.45\columnwidth}
        \includegraphics[width=0.95\columnwidth]{figuras/fastgreedy_probability.pdf}
    \end{subfigure}
    \caption{\label{fig:histos} Distribuci\'on de probabilidad de pertenencia de un nodo a una comunidad 
para los diferentes algoritmos utilizados.}
\end{figure}




\begin{figure}[!ht]
    \centering
    \begin{subfigure}[b]{.45\columnwidth}
        \includegraphics[width=0.95\columnwidth]{figuras/join_proba_fastgreedy-edge_betweeness.pdf}
    \end{subfigure}
    \begin{subfigure}[b]{.45\columnwidth}
        \includegraphics[width=0.95\columnwidth]{figuras/join_proba_fastgreedy-infomap.pdf}
    \end{subfigure}\\
    \begin{subfigure}[b]{.45\columnwidth}
        \includegraphics[width=0.95\columnwidth]{figuras/join_proba_fastgreedy-louvain.pdf}
    \end{subfigure}
    \begin{subfigure}[b]{.45\columnwidth}
        \includegraphics[width=0.95\columnwidth]{figuras/join_proba_infomap-edge_betweeness.pdf}
    \end{subfigure}\\
    \begin{subfigure}[b]{.45\columnwidth}
        \includegraphics[width=0.95\columnwidth]{figuras/join_proba_infomap-louvain.pdf}
    \end{subfigure}
    \begin{subfigure}[b]{.45\columnwidth}
        \includegraphics[width=0.95\columnwidth]{figuras/join_proba_louvain-edge_betweeness.pdf}
    \end{subfigure}
    \caption{\label{fig:matrix} Distribuci\'on de probabilidad conjunta de pertenencia de un nodo a una comunidad 
de cada par de algotirmos.}
\end{figure}

La \textit{Informaci\'on Mutua} total, normalizada, es mostrada en la 
Tabla \ref{tab:mi} de la cual se puede observar que los algoritmos {\it infomap}
y {\it louvain} son los m\'as similares con una semejanza del $86.2\%$,  
las rutinas {\it Fast Greedy} y {\it Edge Betweeness} muestran la menor correlaci\'on
con una similitud del $66.2\%$ mientras que en general el resto coinciden en un rango 
de $70\%-80\%$


\begin{table}[!h]
    \centering
    \caption{\label{tab:mi} Informaci\'on mutua entre las particiones encontradas por cada algoritmo.}
    {\scriptsize
    \begin{tabularx}{0.9\textwidth}{Xl|ccccX}
        \hline\hline
        &                 &  Fast Greedy    & Edge betweeness   & Infomap   & Louvain &  \\
        \hline
        & Fast Greedy     & 1.000           & 0.662             & 0.767     & 0.794 \\
        & Edge Betweeness &                 & 1.000             & 0.771     & 0.732 \\
        & Infomap         &                 &                   & 1.000     & 0.862 \\
        & Louvain         &                 &                   &           & 1.000 \\
        \hline\hline
    \end{tabularx}
    }
\end{table}





\subsection{Relaci\'on de las comunas con g\'enero}
\label{sec:relacion_con_genero}

Para cuantificar la relaci\'on entre las comunas deducidas por los diferentes algoritmos (e.g. {\it greedy}) y el g\'enero, usamos la ec. \ref{eq:info_mutua_norm} identificando a las comunas con $\{C_1\}$ y a las etiquetas de g\'enero con $\{C_2\}$.
%--- ahora veamos nuestro caso
En la figura \ref{fig:prob_conj} mostramos, en el encabezado de cada panel, los valores de la informaci\'on mutua $I_n$, los cuales caen en el intervalo $(0.10 - 0.21)$, es decir que $I_n \ll 1$ en todos los casos; esto nos dice que el conjunto de comunas ($\{C_1\}$) deducido por cierto algoritmo (e.g. {\it greedy}) no nos da mucha informaci\'on sobre el g\'enero ($\{C_2\}$).
%--- test de consistencia
Como test de consistencia para esto  \'ultimo, hicimos sorteos del g\'enero de cada nodo(manteniendo constante el n\'umero total de masculinos y femeninos por separado), y contabilizamos el n\'umero de enlaces entre pares de g\'eneros distintos $n_ig$.
En la figura \ref{fig:hist_sort_sex} mostramos un histograma de $n_ig$, y en l\'inea negra el valor asociado para la red real (original).
De aqui vemos que el valor de la red real esta apartado $\sim 1 \sigma$ del valor medio del histograma; lo cual significa que hay una ligera tendencia a que las comunas tengan muchos ejemplares de un sexo en particular.
Esto \'ultimo es consistente con el bajo valor de $I_n$ discutido mas arriba.


%--- prob conjunta `p12` para c/algoritmo
\begin{figure}
    \centering
    \img{0.5}{p12_greedy}
    \img{0.5}{p12_betweenness}
    \img{0.5}{p12_infomap}
    \img{0.5}{p12_louvain}
    \caption{
    Valores de las matrices de probabilidad conjunto para los algoritmos {\it greedy} (izquierda, arriba) {\it betweenness} (derecha, arriba), {\it infomap} (izquierda, abajo) y {\it louvain} (derecha, abajo). 
    }
\label{fig:prob_conj}
\end{figure}


%--- histograma de sorteos de sexo
\begin{figure}
    \centering
    \img{0.6}{hist_sort_sex}
    \caption{
    Distribuci\'on del n\'umero de enlaces entre g\'eneros diferentes, para diferentes realizaciones de sorteo del sexo de los nodos de la red (manteniendo constante el n\'umero de masculinos y femeninos por separado).
    La l\'inea negra muestra el valor que corresponde a la red original que caracterizamos en este trabajo.
    La zona sombreada en celeste representa la regi\'on que cubre la desviaci\'on est\'andar respecto de la media.
    El valor de la red original (o real) se aparta $\sim 1 \sigma$ respecto del centro de la distribuci\'on, lo cual muestra una ligera tendencia a la existencia de comunas que tienen muchos ejemplares de un sexo en particular.
    Esto es consistente con el bajo valor ($\ll 1$) de la informaci\'on mutua $I_n$ (ver ec. \ref{eq:info_mutua_norm} y Secc. \ref{sec:info_mutua}).
    }
\label{fig:hist_sort_sex}
\end{figure}

%EOF
